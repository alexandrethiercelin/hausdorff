

\documentclass{article}


\usepackage[french]{babel}
\usepackage[T1]{fontenc}
\usepackage[latin1]{inputenc}

\usepackage[babel,french=quotes]{csquotes}

\title{Le jeune chien fou �cumant}
\date{\today}

\begin{document}
\maketitle

Un instant le soleil dans les yeux, l'instant d'apr�s le soleil dans le dos, l'instant encore d'apr�s dans les camps d'ombre jet�s ici et l� au hasard des t�lescopages. Hippias est � la manoeuvre avec Anselm von Bar dans sa nacelle autour de laquelle, escorte tr�s mobile, Alexis et Isidore von Bar n'en finissent pas d'�changer leurs positions. Partout la Hauptstadt �berhaupt se lance � la poursuite de l'oncle et de ses trois neveux telle l'onde qui sur le sable �blouissant s'aplatit pour prendre encore de la vitesse et il faut tout le coup de main magistral d'Hippias pour d�jouer les cabrioles du jeune chien fou �cumant qui, joie pure aboyante, apr�s chaque nouvelle esquive n'en finit pas de revenir sur eux au bout de longues courbes charg� de la puissance des �l�ments �chauff�s par tant de v�locit�. Chaque changement de direction, brusque � la limite de la rupture, en m�me temps qu'il lance l'infatigable poursuivant sur une fausse piste, fait basculer dans un nouveau syst�me de coordonn�es la petite �quip�e et avec elle le soleil et les grands aplats de ciel bleu, les murailles inscrites et les �clats de toits, les trams et les bus, les bicyclettes et autres bolides � pied ou motoris�s, les enseignes et autres affichages publicitaires, et jusqu'� la Fernsehturm, hochet cosmique qu'agite sans m�nagement la petite main d'Anselm von Bar. 

\end{document}