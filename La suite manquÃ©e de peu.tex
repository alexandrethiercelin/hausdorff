
\documentclass{article}


\usepackage[french]{babel}
\usepackage[T1]{fontenc}
\usepackage[latin1]{inputenc}

\usepackage[babel,french=quotes]{csquotes}

\title{La suite manqu�e de peu}
\date{\today}

\begin{document}
\maketitle


Le temps qu'il avait pour la simple et bonne raison qu'il le prenait (� qui? � quoi? � ses poursuivants de le dire!), Hippias Zwaenepoel le prit pour dire ce qu'� l'endroit d'Antoine Zwaenepoel il avait sur le coeur. C'est ce que une premi�re fois in extenso tout haut il dit. Sur le point de passer � la suite soudain il h�sita. Une seconde passa, puis une autre, puis encore une autre. Une minute enfin s'�coula. Alors il se reprit et une seconde fois, cette fois tout bas, in extenso il redit tout ce qu'� l'endroit du m�me Antoine Zwaenepoel il avait sur le coeur. Une nouvelle fois la suite � lui se pr�senta. Mais � nouveau au lieu d'y passer le fils d'Antoine Zwaenepoel h�sita. C'est sans doute alors qu'il s'avisa que ce qu'il venait de dire une premi�re fois in extenso tout haut dans le m�me temps il l'avait dit assis, alors que ce qu'il venait de dire une seconde fois in extenso tout bas dans le m�me temps il l'avait dit debout. C'�tait assez pour laisser une impression d'inachev�. Zwaenepoel le fils se reprit et une troisi�me fois, cette fois assis tout bas, in extenso il dit tout ce qu'� l'endroit de Zwaenepoel le p�re il avait sur le coeur. Il en manquait �videmment une quatri�me et c'est sur elle qu'Hippias embraya presque imm�diatement en sautant sur ses pieds joints pour dire in extenso, cette fois debout tout haut, tout ce qu'� l'endroit d'Antoine il avait sur le coeur. Alors et seulement alors il eut le sentiment triomphal d'avoir �puis� son sujet et de pouvoir enfin passer � autre chose. Mais sa quatri�me fois, sans doute parce que, debout tout haut, ostensiblement d�monstrative donc, elle ne pouvait rester sans r�plique, fut la fois de trop pour Photine von Bar, laquelle avait jusque-l� assist� sans rien dire aux �chafaudages non moins physiques que verbaux de son fr�re. C'est ainsi qu'au lieu de passer � la suite tant convoit�e, Hippias ne put que la regarder une nouvelle fois lui �chapper. 

\medskip

- Mon fr�re, tu es dur mais plus encore tu es injuste avec les parents. As-tu vraiment besoin de ces exag�rations pour te tenir en forme? Je croyais que mon grand fr�re tenait tout seul. Je me serais tromp�e? Papa n'est pas ce que tu dis. 

\end{document}