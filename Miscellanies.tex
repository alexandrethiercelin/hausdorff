
\documentclass{article}


\usepackage[french]{babel}
\usepackage[T1]{fontenc}
\usepackage[latin1]{inputenc}

\usepackage[babel,french=quotes]{csquotes}

\title{Miscellanies}
\date{\today}

\begin{document}
\maketitle


Ils ne pensent pas dans les villes qu'ils habitent (ils ne savent pas ce qu'ils font)

\bigskip

All mathematics is divided into three parts: cryptography (paid for by CIA, KGB and the like), hydrodynamics (supported by manufacturers of atomic submarines) and celestial mechanics (financed by military and other institutions dealing with missiles, such as NASA.). 

Cryptography has generated number theory, algebraic geometry over finite fields, algebra, combinatorics and computers. 

Hydrodynamics procreated complex analysis, partial differential equations, Lie groups and algebra theory, cohomology theory and scientific computing. 

Celestial mechanics is the origin of dynamical systems, linear algebra, topology, variational calculus and symplectic geometry. 

The existence of mysterious relations between all these different domains is the most striking and delightful feature of mathematics (having no rational explanation). (V. I. Arnold, Polymathematics: Is Mathematics a Single Science or a Set of Arts? \textit{Arnold: Swimming Against the Tide}, AMS, 2014, p. 35. Originally published in \textit{Mathematics: Frontiers and Perspectives}, AMS, 2000, pp. 403-416)

\bigskip

The Russian way to formulate problems is to mention the first nontrivial case (in a way that no one would be able to simplify it). The French way is to formulate it in the most general form making impossible any further generalisation. (V. I. Arnold, Polymathematics: Is Mathematics a Single Science or a Set of Arts? \textit{Arnold: Swimming Against the Tide}, AMS, 2014, p. 38. Originally published in \textit{Mathematics: Frontiers and Perspectives}, AMS, 2000, pp. 403-416)

\bigskip

Les religions ont des avantages pratiques qu'on ne saurait sous-estimer. Des dieux peuvent �tre emport�s partout ou peuvent se retrouver partout. Une civilisation nomade, ou qui se repr�sente comme telle, afin de rester l�g�re donnera dans la religion qui permet � tr�s peu de frais de se retrouver chez soi partout. On peut difficilement reprocher aux nomades d'avoir un commerce exag�r� avec les images. Les images sont ce que l'on emporte avec soi quand ce dont elles sont les images a disparu. C'est comme une alternative entre deux formes de civilisation: ou bien sans religion mais lourdement �quip�, ou bien l�ger mais avec religion. Les uns se connectent avec des objets, les autres avec leurs dieux. 

\bigskip

L'�cran dans toutes les mains, dans celles du s�dentaire comme dans celles du nomade, dans celle du sur�quip� comme dans celle du d�muni. 


\end{document}

